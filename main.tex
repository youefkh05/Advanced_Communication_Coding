\UseRawInputEncoding

\documentclass[12pt,a4paper]{report}

% ==================================================
% Packages
% ==================================================
\usepackage{graphicx}
\usepackage{amsmath,amssymb}
\usepackage{geometry}
\usepackage{setspace}
\usepackage{caption}
\usepackage{float}
\usepackage{booktabs}
\usepackage{hyperref}
\usepackage{xcolor}
\usepackage{listings}

\geometry{margin=1in}
\onehalfspacing

% ==================================================
% MATLAB Listings Style (Colorful, Clean)
% ==================================================
\definecolor{matlabblue}{rgb}{0,0,1}
\definecolor{matlabgreen}{rgb}{0,0.5,0}
\definecolor{matlaborange}{rgb}{1,0.4,0}
\definecolor{matlabgray}{rgb}{0.4,0.4,0.4}

\lstdefinestyle{MATLABstyle}{
    language=Matlab,
    basicstyle=\ttfamily\small,
    keywordstyle=\color{matlabblue}\bfseries,
    commentstyle=\color{matlabgreen},
    stringstyle=\color{matlaborange},
    numberstyle=\tiny\color{matlabgray},
    numbers=left,
    stepnumber=1,
    numbersep=8pt,
    frame=single,
    breaklines=true,
    showstringspaces=false,
    tabsize=4
}

\lstset{style=MATLABstyle}

% ==================================================
\begin{document}
% ==================================================

% ==================================================
% Cover Page
% ==================================================
\begin{titlepage}
\centering
\vspace*{1cm}

{\Large \textbf{Electronics and Electrical Communications Engineering Department}}\\
{\Large \textbf{Faculty of Engineering – Cairo University}}\\[1.2cm]

{\Huge \textbf{Information Theory Project – Part II}}\\
{\LARGE \textbf{Channel Coding}}\\[1cm]

{\Large ELC4020 – Advanced Communication Systems}\\
{\Large 4\textsuperscript{th} Year – 1\textsuperscript{st} Semester}\\
{\Large Academic Year 2025/2026}\\[1.2cm]

\textbf{Prepared by:}\\[0.3cm]
\begin{tabular}{ccc}
\textbf{Name} & \textbf{Section} & \textbf{ID} \\ \midrule
Mohamed Ahmed Abd El Hakam & 3 & 9220647 \\
Yousef Khaled Omar Mahmoud & 4 & 9220984 \\
Ahmed Wagdy Mohy Ibrahim & 1 & 9220120 \\
Abdelrahman Essa Elsayed & 2 & 9220469 \\
\end{tabular}

\vfill

\textbf{Instructor:} Eng. Mohamed Khaled\\
\textbf{Supervised by:} Dr. Mohamed Nafie, Dr. Mohamed Khairy

\end{titlepage}

% ==================================================
% Front Matter
% ==================================================
\pagenumbering{roman}
\tableofcontents
\listoffigures
\newpage
\pagenumbering{arabic}

% ==================================================
% Executive Summary
% ==================================================
\chapter*{Executive Summary}
\addcontentsline{toc}{chapter}{Executive Summary}

This report investigates the effect of channel coding techniques on the bit error rate (BER) performance of digital communication systems operating over an additive white Gaussian noise (AWGN) channel. Different modulation schemes and coding strategies are evaluated to highlight the trade-offs between reliability, energy efficiency, and spectral efficiency.

% ==================================================
\chapter{Introduction}
Channel coding is used to introduce controlled redundancy in transmitted data to combat the effect of noise and channel impairments. In this project, several channel coding techniques are analyzed and compared using BER as the main performance metric. All simulations are implemented using MATLAB.

% ==================================================
\chapter{Simulation Environment}
All experiments are implemented using MATLAB and integrated into a unified graphical user interface (GUI). 
The GUI allows the user to select any problem from Question 3 to Question 9, execute the corresponding simulation, 
and automatically display numerical outputs and generated figures.

\begin{figure}[H]
\centering
\includegraphics[width=0.75\textwidth]{figures/GUI_Main_Window.png}
\caption{Graphical User Interface used to execute channel coding experiments.}
\end{figure}

% ==================================================
\chapter{Problem 3: Uncoded BPSK over AWGN}

The theoretical BER of uncoded BPSK over an AWGN channel is given by:
\[
\text{BER}_{\text{BPSK}} = \frac{1}{2}\,\mathrm{erfc}\left(\sqrt{\frac{E_b}{N_0}}\right)
\]

\begin{figure}[H]
\centering
\includegraphics[width=0.85\textwidth]{figures/Q3_Uncoded_BPSK_AWGN.png}
\caption{BER performance of uncoded BPSK over an AWGN channel.}
\end{figure}

\section*{Discussion}
This experiment evaluates the BER performance of uncoded BPSK over an AWGN channel 
for $E_b/N_0$ ranging from $-3$ to $10$ dB while transmitting 110000 information bits.
The simulated BER closely matches the theoretical expression, confirming the validity 
of the noise model and modulation process.

As expected, increasing $E_b/N_0$ results in a significant reduction in BER due to lower noise variance.
This experiment serves as a baseline reference for all subsequent coded systems.
\textbf{This discussion answers Question 3.}

\section*{MATLAB Code}
\lstinputlisting[
    caption={MATLAB implementation of uncoded BPSK over AWGN},
    label={lst:Q3}
]{code/Q3.txt}




% ==================================================
\chapter{Problem 4: Repetition-3 Coded BPSK (Hard Decision)}

This experiment evaluates repetition-3 coding under two scenarios:
\begin{itemize}
\item Same energy per transmitted bit
\item Same energy per information bit
\end{itemize}

\begin{figure}[H]
\centering
\includegraphics[width=0.85\textwidth]{figures/Q4_BPSK_Repetition3_HardDecision.png}
\caption{BER comparison of uncoded BPSK and repetition-3 coded BPSK using hard decision decoding.}
\end{figure}

\section*{Discussion}
Repetition-3 coding with hard decision decoding is evaluated under two energy constraints.
When the same energy per transmitted bit is used, the BER is significantly improved due to redundancy 
and majority voting, at the cost of increased total transmitted energy.

When the same energy per information bit is maintained, the BER improvement is reduced since the 
energy is divided among repeated bits.
Such schemes are suitable for power-limited systems with relaxed bandwidth constraints.
\textbf{This discussion answers Question 4 (a) and (b).}



\section*{MATLAB Code}
\lstinputlisting[
    caption={MATLAB implementation of repetition-3 coded BPSK (hard decision)},
    label={lst:Q4}
]{code/Q4.txt}

% ==================================================
\chapter{Problem 5: Repetition-3 Coded BPSK (Soft Decision)}

Soft-decision decoding improves BER performance by utilizing amplitude information rather than binary decisions.

\begin{figure}[H]
\centering
\includegraphics[width=0.85\textwidth]{figures/Q5_BPSK_Repetition3_SoftDecision.png}
\caption{BER performance of repetition-3 coded BPSK using soft decision decoding.}
\end{figure}

\section*{Discussion}
Soft decision decoding further improves BER performance by exploiting received signal amplitudes 
instead of binary decisions.
Compared to hard decision decoding, a clear coding gain is observed, especially at low $E_b/N_0$.

This approach is preferred in systems where receiver complexity is acceptable and higher reliability 
is required, such as modern wireless and satellite communication systems.
\textbf{This discussion answers Question 5.}


\section*{MATLAB Code}
\lstinputlisting[
    caption={MATLAB implementation of repetition-3 coded BPSK (soft decision)},
    label={lst:Q5}
]{code/Q5.txt}

% ==================================================
\chapter{Problem 6: Hamming (7,4) Coded BPSK}

The Hamming (7,4) code has a minimum distance of $d_{\min}=3$, allowing single-bit error correction.

\begin{figure}[H]
\centering
\includegraphics[width=0.85\textwidth]{figures/Q6_(7,4)_Hamming_code.png}
\caption{BER comparison between uncoded BPSK and Hamming (7,4) coded BPSK.}
\end{figure}

\section*{Discussion}
The Hamming (7,4) code introduces structured redundancy and enables single-bit error correction.
The minimum distance of this code is $d_{\min}=3$, which allows correction of one error per codeword.

The results show a noticeable BER improvement compared to uncoded BPSK when using the same energy per 
information bit.
This code is recommended when reducing BER is the primary objective and transmission time is not critical.
\textbf{This discussion answers Question 6 (c) and (d).}


\section*{MATLAB Code}
\lstinputlisting[
    caption={MATLAB implementation of Hamming (7,4) coded BPSK},
    label={lst:Q6}
]{code/Q6.txt}

% ==================================================
% ==================================================
\chapter{Problem 7: Hamming (15,11) with BPSK and QPSK}

In this problem, the performance of a Hamming $(15,11)$ coded system is evaluated using both BPSK and QPSK modulation schemes. The objective is to study the impact of modulation choice on bit error rate (BER) performance and transmission time under the same energy constraints.

\begin{figure}[H]
\centering
\includegraphics[width=0.85\textwidth]{figures/Q7_QPSK_BPSK_(15,11)_Hamming code.png}
\caption{BER performance comparison between uncoded BPSK and the proposed QPSK with Hamming (15,11) coding.}
\end{figure}

\section*{Discussion}

The Hamming $(15,11)$ code has a higher code rate compared to the $(7,4)$ Hamming code, which results in reduced redundancy and improved spectral efficiency. When combined with BPSK modulation, the coded system achieves a noticeable BER improvement compared to uncoded transmission.

However, the use of channel coding increases the number of transmitted bits, which in turn increases the transmission time when using BPSK modulation. To address this issue, QPSK modulation is proposed as an alternative.

\textbf{Answer to Question 7(e):}  
The use of a Hamming $(15,11)$ code is recommended for reducing the bit error rate when transmission time is not a critical constraint, as it provides effective error correction with moderate redundancy.

\textbf{Answer to Question 7(f):}  
To keep the transmission time equal to or less than that of uncoded BPSK (Problem 3), QPSK modulation is proposed in combination with the Hamming $(15,11)$ code. Since QPSK transmits two bits per symbol, it compensates for the redundancy introduced by channel coding and maintains the overall transmission time.

\textbf{Answer to Question 7(g):}  
The plotted BER curve of the proposed scheme (QPSK with Hamming $(15,11)$ coding) demonstrates that it achieves nearly the same BER performance as coded BPSK while maintaining equal or shorter transmission time. This confirms that the proposed scheme provides improved reliability without sacrificing transmission efficiency.

\textbf{This discussion fully answers Question 7 (e), (f), and (g).}


\section*{MATLAB Code}
\lstinputlisting[
    caption={MATLAB implementation of Hamming (15,11) coded BPSK and QPSK},
    label={lst:Q7}
]{code/Q7.txt}

% ==================================================
\chapter{Problem 8: BCH-Coded 16-QAM vs Uncoded QPSK}

This problem compares spectral efficiency and coding gain using BCH(255,131) with 16-QAM.

\begin{figure}[H]
\centering
\includegraphics[width=0.85\textwidth]{figures/Q8_QPSK_vs_16QAM_BCH.png}
\caption{BER comparison between uncoded QPSK and BCH(255,131) coded 16-QAM.}
\end{figure}

\section*{Discussion}
This experiment compares uncoded QPSK with BCH(255,131) coded 16-QAM for the transmission 
of 26.2 million bits over an AWGN channel.
Although 16-QAM is inherently more sensitive to noise, the strong BCH coding provides 
significant coding gain.

At moderate and high $E_b/N_0$, the coded 16-QAM system outperforms uncoded QPSK while 
offering higher spectral efficiency.
This makes it suitable for high-data-rate communication systems.
\textbf{This discussion answers Question 8 (h) and (i).}


\section*{MATLAB Code}
\lstinputlisting[
    caption={MATLAB implementation of QPSK vs BCH-coded 16-QAM},
    label={lst:Q8}
]{code/Q8.txt}

% ==================================================
\chapter{Problem 9: Convolutional Encoding}

A convolutional encoder with rate $2/3$ is implemented, and the state transition table is visualized.

\begin{figure}[H]
\centering
\includegraphics[width=0.85\textwidth]{figures/Q9_Convolutional_Encoder_Table.png}
\caption{Convolutional encoder state transition and output table.}
\end{figure}

\section*{Discussion}
A convolutional encoder with rate $2/3$ is implemented using the specified generator sequences.
The encoder state transitions and outputs are visualized in a table, illustrating the memory-based 
nature of convolutional coding.

Such encoders are widely used in practical communication systems due to their continuous encoding 
structure and strong error-correcting capability when combined with optimal decoding algorithms.
\textbf{This discussion answers Question 9.}


\section*{MATLAB Code}
\lstinputlisting[
    caption={MATLAB implementation of convolutional encoder},
    label={lst:Q9}
]{code/Q9.txt}

% ==================================================
\chapter{Conclusion}

Channel coding significantly enhances system reliability. Strong coding combined with higher-order modulation enables higher data rates without sacrificing BER performance. QPSK and BCH-coded 16-QAM demonstrate superior spectral efficiency compared to uncoded systems.

% ==================================================
\appendix
\chapter{ GUI Implementation}

\lstinputlisting[
    caption={MATLAB GUI for Channel Coding Project},
    label={lst:GUI}
]{code/GUI.txt}

% ==================================================
\end{document}
